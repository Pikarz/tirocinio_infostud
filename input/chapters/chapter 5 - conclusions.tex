% !TEX root = ../thesis.tex
    
%**************************************************************
% Conclusioni
%**************************************************************

\chapter{Conclusioni: un passo avanti verso soluzioni migliori}
        Nell'ambito degli studi affrontati, sono stati esplorati diversi algoritmi per la rilevazione di anomalie 
        sulla serie temporale multivariata estratta dalla piattaforma indispensabile per tutta la nostra comunità accademica. 
        Gli esperimenti hanno fornito preziose informazioni sulle prestazioni di 
        questi algoritmi e sulle loro potenzialità in questo complesso e affascinante dominio di applicazione; tali soluzioni,
        auspicabilmente, potrebbero essere analizzate per far sì che vi siano miglioramenti sulle operazioni e sul 
        tempo di risposta e di processamento dei servizi richiesti, potenzialmente garantendo un ambiente migliore a tutti 
        coloro che, giornalmente, beneficiano della piattaforma universitaria Infostud. 
        È stato un onore per il sottoscritto poter interagire con dati di tale rilevanza e di così
        stretto coinvolgimento.

        \paragraph{ARMA: il modello consolidato} Nonostante ARMA\cite{arma} (Auto Regressive Moving Average) non sia stato 
        originariamente progettato per il rilevamento di anomalie, rimane un pilastro tra gli algoritmi tradizionali. 
        ARMA è un modello robusto ed efficace, basato su semplici metodi statistici ma adattabile a numerosi contesti della 
        data science, e dimostra solidità anche nella rilevazione di anomalie.

        \paragraph{OC-SVM: una soluzione sfacciata} L'algoritmo dall'approccio metodologico del machine learning visto 
        in questi studi è stato OC-SVM\cite{ocsvm}. Sebbene questo algoritmo fosse molto sfavorito per il contesto
        preso in considerazione, visto l'intrinseco sbilanciamento tra le classi nell'anomaly detection,
        ha comunque offerto una soluzione degna di interesse. OC-SVM è un algoritmo flessibile e ci ha garantito
        un'ottima base su cui misurare le performance future.


        \paragraph{Telemanom: la non-soluzione} Purtroppo Telemanom\cite{telemanom}, nonostante sia un modello molto recente 
        che si focalizza sull'anomaly detection attraverso il deep learning, offre una soluzione che overfitta i dati. 
        L'overfit non è tanto dovuto agli iperparametri, quanto a come il modello gestisce le anomalie e a come 
        esse sono naturalmente distribuite nel dataset preso in analisi. La soluzione di Telemanom, seppur buona numericamente, 
        non è da prendere in considerazione.

        \paragraph{MSCRED: un promettente avanzamento} MSCRED\cite{mscred} (Multi-Scale Convolutional Recurrent Encoder-Decoder) 
        è un modello relativamente giovane che si basa sul deep learning. Le analisi illustrate suggeriscono che MSCRED, grazie 
        alla sua capacità di catturare l'intercorrelazione tra i segnali e la loro dipendenza temporale, emerge come un 
        candidato di spicco per futuri sviluppi nel campo dell'anomaly detection. Il modello ha dimostrato ottime 
        performance senza richiedere un eccessivo tuning degli iperparametri, lasciando intravedere un promettente futuro.

        \paragraph{Riassunto: Le Performance degli Algoritmi} La \hyperref[tab:recap-perf]{Tabella 5.1.} riassume 
        le performance dei vari algoritmi esaminati mediante le metriche principali prese in considerazione.


    \begin{table}[H]
        \centering
        \caption{Performance dei modelli analizzati}
        \begin{tabular}{lcccc}
        \toprule
        \textbf{Modello} & \textbf{Precisione} & \textbf{Recall} & \textbf{F1-score} \\
        \midrule
        \textbf{OC-SVM} & 0.331 & 0.290 & 0.309 \\
        \textbf{ARMA} & 0.332 & 0.248 & 0.284 \\
        \textbf{Telemanom}\footnote{}  & \emph{1.0} & \emph{0.5} & \emph{0.66} \\
        \textbf{MSCRED} & \textbf{0.711} & \textbf{0.857} & \textbf{0.777} \\
        \bottomrule
        \end{tabular}
        \label{tab:recap-perf}
    \end{table}
    
        \footnotetext{La soluzione non è valida perché il modello non è adatto al contesto di InfoSapienza}
        \paragraph{Il futuro dell'anomaly detection} L'obiettivo per il futuro che ci attende è quello di sviluppare 
        soluzioni sempre più accurate, che possano identificare le anomalie nei sistemi complessi attribuendo un valore 
        numerico che identifichi lo stato, anomalo o meno, del sistema. Ciò garantirà non solo la possibilità di prendere atto 
        prontamente delle anomalie, ma renderà possibile anche la loro prevenzione, contribuendo così a migliorare 
        la sicurezza e minimizzando i periodi di down dei sistemi in una varietà di settori industriali, in cui spesso 
        vi è in ballo anche la vita umana.
