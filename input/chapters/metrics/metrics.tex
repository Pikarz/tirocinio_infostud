%**************************************************************
% Metriche di Valutazione
%**************************************************************

Nel contesto dell'analisi condotta sui dati osservati della piattaforma Infostud, è fondamentale valutare numericamente 
le soluzioni dei modelli applicati ai fini della valutazione delle loro performance. A tale scopo, vengono utilizzate 
tre metriche importanti: la precisione (precision), il richiamo (recall) e la F1-score (F1) al fine di misurare 
l'efficacia delle soluzioni in modo rigoroso. Queste metriche sono essenziali per comprendere in modo completo ed 
esaustivo l'efficacia degli approcci metodologici applicati.

\subsection{Precisione (Precision)}\label{precision}
    La precisione è una metrica che misura quanto un modello è accurato nel predire gli esempi positivi, o anomali nel 
    contesto dell'anomaly detection, rispetto a tutti 
    i casi che il modello ha classificato come positivi. 
    Per calcolare la precisione, viene valutata la frazione di predizioni positive fatte dal modello che sono effettivamente 
    corrette.
    \[Precision = \frac{TP}{TP+FP}\]\
    Dove:
    \begin{enumerate}
        \item TP (True Positives) rappresenta il numero di casi positivi, o anomali,
        correttamente classificati dal modello.
        \item FP (False Positives) sono i casi negativi erroneamente classificati come positivi, o anomali, dal modello.
    \end{enumerate}
    La precisione fornisce un riscontro numerico che, se massimizzato, rende il modello affidabile quando afferma 
    che un caso è positivo. Tuttavia, la precisione da sola potrebbe non fornire una visione completa delle prestazioni 
    di un modello.

\subsection{Richiamo (Recall)}
    Il richiamo misura la capacità di un modello di individuare tutti gli esempi positivi correttamente. In altre parole, 
    indica quanto il modello fornisce soluzioni che correttamente identificano gli esempi positivi. Il richiamo 
    viene calcolato attraverso la frazione di predizioni positive, o anomale, fatte 
    dal modello rispetto al totale dei casi positivi effettivi.
    \[Recall = \frac{TP}{TP+FN}\]
    Dove:
    \begin{enumerate}
        \item TP sono i True Positives, definiti nella \hyperref[precision]{Sottosezione 4.1.1}
        \item FN (False Negatives) rappresenta il numero di casi positivi, o anomali, erroneamente classificati come negativi 
              dal modello.
    \end{enumerate}
    Un alto valore di richiamo indica che il modello ha un'ottima capacità di individuare gli esempi positivi, ma 
    non tiene in considerazione il numero di casi falsi positivi.

\subsection{F1-Score (F1)}\label{f1-score}
    L'F1-score è una metrica che combina la precisione e il richiamo in un valore che tiene conto sia dei falsi positivi 
    che dei falsi negativi. La metrica bilancia la precisione e il richiamo. La formula per calcolare l'F1-score è 
    la seguente:
    \[F1 = \frac{2\cdot Precision \cdot Recall}{Precision+Recall}\]
    L'F1-score, quindi, mira a cercare un compromesso tra i valori di precisione e richiamo.

