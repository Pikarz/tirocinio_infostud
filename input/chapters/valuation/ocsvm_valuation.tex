%**************************************************************
% Valutazione OC-SVM
%**************************************************************

Il metodo che si basa sul machine learning, One-Class Support Vector Machine, arricchisce l'analisi e lo studio 
delle anomalie sul daset di Infostud fornendo una soluzione molto buona sull'insieme di validazione con uno score F1 pari 
a 0.716, e una discreta sull'insieme di test, come mostrato nella 
\hyperref[tab:ocsvm-metrics]{Figura 4.2.}


\begin{table}[H]
    \centering
    \caption{Soluzione OC-SVM.}
    \begin{tabular}{lr}
    \toprule
    \textbf{Soluzione OC-SVM sul test set}  \\
    \midrule
    \multirow{3}{*}{\textbf{Metriche}} 
        & Precisione: 0.331 \\
        & Recall: 0.290 \\
        & F1-score: 0.309 \\
    \bottomrule
    \end{tabular}
    \label{tab:ocsvm-metrics}
\end{table}

I risultati mostrano che OC-SVM è in grado di produrre buoni risultati durante la fase di 
validazione ma fallisce durante la fase di test. Tuttavia, è importante notare che il modello è stato applicato 
in un contesto avverso, caratterizzato da un forte sbilanciamento tra le classi. Nonostante l'ambiente sfavorevole, 
questa soluzione è un punto di riferimento ragionevole per valutare le prestazioni del modello MSCRED.