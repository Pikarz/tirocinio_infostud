%**************************************************************
% Valutazione ARMA
%**************************************************************

Come è stato già osservato, il modello statistico ARMA racchiude in modo discreto il comportamento del dataset e indica come anomalie i casi in 
cui la previsione del modello si discosta molto dai dati reali. In particolare, gli iperparametri scelti 
hanno riportato uno score F1 di 0.750 sul dataset di validation, mentre 
le metriche osservate sull'insieme di test sono riportate nella \hyperref[tab:arma-metrics]{Tabella 4.1.}

\begin{table}[H]
    \centering
    \caption{Risultati ARMA.}
    \begin{tabular}{lr}
    \toprule
    \textbf{Soluzione ARMA sul test set}  \\
    \midrule
    \multirow{3}{*}{\textbf{Metriche}} & Precisione: 0.332 \\
    & Recall: 0.248 \\
    & F1-score: 0.284 \\
    \bottomrule
    \end{tabular}
    \label{tab:arma-metrics}
\end{table}

I risultati mostrano che ARMA riesce a catturare la tendenza generale del modello e molti 
andamenti anomali vengono predetti come tali. La soluzione non è ottimale, ma fornisce 
comunque una buona base per la valutazione di MSCRED.