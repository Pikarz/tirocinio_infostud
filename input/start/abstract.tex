% !TEX root = ../thesis.tex
    
%**************************************************************
% Abstract
%**************************************************************

\begin{abstract}
    Le serie temporali multivariate rappresentano un campo di studio 
    di notevole rilevanza in svariati sistemi distribuiti in tutto 
    il mondo. In un certo senso, queste serie di dati delineano 
    il ciclo di vita del sistema. Si può dunque dedurre che 
    attraverso le serie temporali vengono modellati sia i periodi 
    in cui il sistema funziona normalmente, che dualmente i periodi
    più "anomali".

    L'analisi delle serie temporali multivariate, con l'obiettivo 
    di identificare e predire anomalie, non deve solamente tener 
    conto della dipendenza temporale, un problema già complesso di 
    per sé, ma deve anche considerare la correlazione tra i vari 
    segnali all'interno del sistema. 
    
    Nel corso degli anni, diversi algoritmi sono stati impiegati 
    per l'analisi di anomalie nelle serie temporali, tra cui 
    SVM\cite{ocsvm} (Support Vector Machine), ma con risultati insoddisfacenti.
    Questi modelli presentano una limitazione fondamentale 
    che compromette la loro efficacia nell'analisi delle serie 
    temporali: la loro incapacità di catturare la correlazione 
    temporale, una caratteristica essenziale per condurre 
    un'analisi accurata di tali serie. Modelli come ARMA\cite{arma}, invece,
    riescono a catturare la dipendenza temporale, ma sono per natura 
    monovariati, per cui non sono in grado di osservare l'intercorrelazione 
    dei segnali.

    La ricerca sull'individuazione delle anomalie, negli anni recenti, ha fatto 
    molti passi avanti. Algoritmi come Telemanom\cite{telemanom} vengono applicati 
    a sistemi reali, complessi e delicati con ottimi risultati.
    
    L'anno 2019 è stato di particolare interesse per l'analisi di anomalie
    di serie temporali multivariate, grazie alla pubblicazione 
    dell'articolo scientifico sul modello Multi-Scale Convolutional 
    Recurrent Encoder-Decoder\cite{mscred} (MSCRED).
    MSCRED affronta la necessità di misurare non solo 
    l'intercorrelazione tra i dati, ma anche la loro dipendenza 
    temporale.

    In questa tesi, verrà applicato MSCRED su un sistema reale che riveste un ruolo di 
    straordinaria importanza per tutta la comunità accademica della Sapienza: il 
    prezioso dataset di Infostud, piattaforma che rappresenta una risorsa inestimabile 
    per la nostra istituzione, essendo un punto di riferimento fondamentale per 
    le attività accademiche e amministrative. 
    
    Verranno valutate le prestazioni di MSCRED, confrontando i risultati ottenuti 
    con quelli generati dagli algoritmi OC-SVM, ARMA e Telemanom. È importante 
    sottolineare che MSCRED, grazie alla sua capacità di analisi delle serie 
    temporali multivariate, si prevede porterà a risultati notevolmente 
    superiori rispetto agli altri modelli. Questa analisi rappresenta 
    un passo significativo verso un possibile miglioramento delle operazioni e 
    dell'efficienza del sistema Infostud.
\end{abstract}